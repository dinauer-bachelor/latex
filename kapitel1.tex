\section{Einleitung in das Wissensmanagement durch KI}
\subsection{Einleitung}
Heutzutage ist die Analyse von Daten in vielen Unternehmen wichtiger als je zuvor und die Gewinnung von Wissen aus Daten gewinnt stetig an Stellenwert. Passend zu dieser Beobachtung schreibt das Marktforschungs- und Beratungsunternehmen Gartner: "Graph analysis is possibly the single most effective competitive differentiator for organizations pursuing data-driven operations and decisions". (Gartner, 2014, zitiert in Robinson, 2015, S. 2) \\
Durch die Einführung einer Künstlichen Intelligenz, um das Wissensmanagement im Unternehmen zu verbessern ergeben sich mehrere Teilaspekte im Umgang mit Wissen. Neben der Möglichkeit, durch das Transformieren von bereits vorhandenem Wissen, um neues Wissen zu kreieren, kann eine Künstliche Intelligenz auch den Prozess des Abfragens und Extrahierens von Wissen aus einer großen Sammlung an Daten unterstützen. Diese Arbeit beschäftigt sich primär mit dem Aspekt der Wissensabfrage und dem Aufbau der hierfür notwendigen Datenstrukturen. (vgl. Jarrahi, 2023, S. 2) \\
Durch die Entwicklung und Bedeutung von Wissensgraphen in der heutigen Zeit sowie die revolutionären Entwicklungen im Bereich der künstlichen Intelligenz ergeben sich für Unternehmen neue Möglichkeiten, das Wissensmanagement in ihrem Betrieb neu zu gestalten. Ziel dieser Arbeit ist es, ein System zu entwerfen und zu implementieren, welches die Erstellen eines Wissensgraphen ermöglicht, der speziell für die Integration mit einem Chatbot, insbesondere einem Large Language Model (LLM), konzipiert ist.\\
Um Chatbots im Unternehmenskontext einsetzen zu können, müssen diese mit dem Unternehmensumfeld, also den Daten und Abläufen des Unternehmens vertraut gemacht werden. Dazu wird untersucht, wie ein Knowledge Graph aus einer bereits existierenden und stetig weiter wachsenden Sammlung aus Tickets eines Jira-Systems erstellt werden kann. Berücksichtigt werden hierbei Duplikate, Dateninkonsistenzen, sowie die Historisierung und Auditierung. Der Hauptaspekt dieser Arbeit ist das Design, sowie die Implementierung des Datenbankschemas und der Extraktor-Komponente. Am Ende wird geprüft, ob sich eine Graphdatenbank für die Integration eines Chatbots eigenet, um möglichst schnell und präsize das Informationsbefürdnis eines Anwender zu befriedigen.\\
Außerdem werden zusätzlich zu den Herausforderungen beim Aufbau des Graphen, mögliche Herausforderungen in Verbindung mit der Data Warehouse Architektur behandelt. Teil dessen sind beispielsweise die Terminierung der Extraktion, das Speichern der Metadaten sowie die Konfiguration des Extraktors, der das extrahieren der Jira Daten vornimmt. \\
\subsection{Vorgehensweise}
Die Struktur dieser Arbeit folgt dem Ansatz des Wasserfallmodells in der Softwareentwicklung. Zu Beginn erfolgt die Anforderungsanalyse. Dabei soll festgelegt werden, welche funktionlen und nicht-funktionalen Anforderungen unser System erfüllen muss und welche optional erfüllt werden sollen. In Folge des Anforderungsmanagements müssen alle Bestandteile des Systems, wie z. B. Datenbankmodelle, Datenbanken sowie Softwarekomponenten mit Applikationslogik identifiziert werden, welche notwendig sind, um den Prozess zum Erstellen des Knowledge Graphen durchzuführen. Anschließen soll dargelegt und begründet werden, für welche konkreten Ausführungen, Implementierungen und Technologien entschieden wird. Es sollen alle signifikanten Vor- und Nachteile erläutert werden. Es soll beschrieben, warum sich diese Technologien gut für eine Integration eignen. Sobald alle Bestandteile des Systems festgelegt sind, wird mit der Planung, dem Design und der Implementierung des Systems begonnen. In dieser Phase werden alle Jira-Objekte erfasst und als Datenbankschema modelliert. Es folgt eine genaue Beschreibung, wie beispielsweise ein Datenbankschema modelliert wird. Außerdem wird beschrieben, wie die ermittelten Anforderungen in Form von Code umgesetzt werden. Bei der Implementierung des Codes sollen zentrale Aspekte und Abschnitte im Code aufgegriffen und genauer erläutert werden. Anschließend wird die Integration sowie die Einführung der Software erläutert. Zuletzt soll geprüft werden, wie eine Künstliche Intelligenz möglichst effizient mit diesem Knowledge Graphen als Datengrundlage integriert werden kann.\\